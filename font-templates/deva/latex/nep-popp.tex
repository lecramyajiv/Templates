%
% Copyright 2023 Lecram Yajiv

% Licensed under the Apache License, Version 2.0 (the "License");
% you may not use this file except in compliance with the License.
% You may obtain a copy of the License at

% http://www.apache.org/licenses/LICENSE-2.0

% Unless required by applicable law or agreed to in writing, software
% distributed under the License is distributed on an "AS IS" BASIS,
% WITHOUT WARRANTIES OR CONDITIONS OF ANY KIND, either express or implied.
% See the License for the specific language governing permissions and
% limitations under the License.
% translated from wikipedia about cheras
% % https://en-m-wikipedia-org.translate.goog/wiki/Chera_dynasty?_x_tr_sl=auto&_x_tr_tl=ml&_x_tr_hl=en-GB


\RequirePackage[orthodox]{nag}
\documentclass[a4paper,12pt,oneside,final]{article}
\usepackage{microtype}
\usepackage[a4paper, margin=1.3cm, nohead, nofoot]{geometry}
\usepackage[T1]{fontenc}
\usepackage{fontspec}
\usepackage{setspace}
\pagestyle{empty}
\onehalfspacing
\usepackage[hindi,provide=*]{babel}
\babelfont[hindi]{rm}[Renderer=Harfbuzz]{Poppins}
\begin{document}
देवनागरी अथवा नागरी बायाँ देखि दायाँ सम्म लेखिने अक्षरात्मक लिपिमा आधारित एक प्राचीन ब्राह्मी लिपि हो। यसको प्रयोग भारतीय उपमहाद्वीपमा हुने गर्दछ। यसको विकास प्राचीन भारतमा पहिलो देखि चौथो शताब्दीको बीचमा भएको थियो भने यसको प्रयोग सातौँ शताब्दी देखी हुँदै आएको छ। देवनागरी लिपिमा, ४७ वटा वर्णहरू हुन्छन् जसमा १४ वटा स्वरवर्ण र ३३ वटा व्यञ्जनवर्ण हुन्छन्। यो संसारमा सबैभन्दा बढी प्रयोग हुने लिपि मध्ये चौथो हो जहाँ यस लिपिमा १२० भन्दा बढी भाषाहरू लेखिन्छ। यो लिपि नेपाल र भारत गणतन्त्रको आधिकारिक लिपिहरू मध्ये एक हो।

यस लिपिको वर्ण विन्यासविज्ञानले भाषाको उच्चारणलाई दर्शाउँदछ। रोमन वर्णमालाको विपरीत, यसमा ठुलो र सानो वर्ण भने हुँदैन। यो बायाँबाट दायाँ सम्म लेखिन्छ। यसमा वर्गको रूपरेखाहरूको भित्र सममित गोलाकार आकारका लागि एक बलियो प्राथमिकता रहेको हुन्छ जुन एक क्षैतिज रेखाद्वारा पहिचान योग्य हुन्छ जसलाई शिरोरेखा भनिन्छ जुन पूर्ण अक्षरहरूका लागि शीर्षको साथ चल्दछ अर्थात् प्रत्येक शब्दमाथिबाट खिचिने रेखा (केही वर्णहरूको माथिबाट रेखा हुँदैन) नै शिरोरेखा हो। झट्ट हेर्दा, अन्य ब्राह्मी लिपिहरू जस्तै बङ्गाली-असमिया, ओडिया वा गुरुमुखी भन्दा फरक देखिन्छ। तथापि, ध्यान पूर्वक यी लिपिहरू हेर्ने हो भने कोण र संरचनात्मक जोड बाहेक यिनीहरूमा निकै समानता हुन्छ। यसमा अक्षरहरूको क्रम व्यवस्था (विन्यास) वैज्ञानिक छ। यसमा स्वर-व्यञ्जन, कोमल-कठोर, अल्पप्राण-महाप्राण, अनुनासिक्य-अन्तस्थ-उष्म इत्यादि वर्गीकरण हुन्छ। उर्दु बाहेक, भारत तथा एसियाका अनेक लिपिहरूको सङ्केत देवनागरीबाट अलग छ। देवनागरी लेखनको दृष्टिबाट सरल, सौन्दर्यको दृष्टिबाट सुन्दर र वाचनको दृष्टिबाट सुपाठ्य छ।

मराठी, पाली, संस्कृत (संस्कृतको लागि प्राचीन नागारी लिपिमा दुई अतिरिक्त व्यञ्जन वर्ण थिएँ), हिन्दी, नेपाली, शेर्पा, प्राकृत, अपभ्रंश, अवधी, भोजपुरी, ब्रज भाषा, छत्तीसगढी, हरियाणवी, मगही, नागपुरी, राजस्थानी, भिली, डोगरी, मैथिली, कश्मीरी, कोङ्कणी, सिन्धी, बोडो, नेवारी, पन्जाबी, मुन्डारी र सन्थाली लगायतका भाषाहरूले देवनागरी लिपि एकमात्र लिपि वा दोस्रो लिपिको रूपमा प्रयोग गर्ने गर्दछन्। देवनागरी लिपि नन्दीनागरी लिपिसँग पनि मिल्दोजुल्दो रहेको छ जुन दक्षिण भारतका विभिन्न प्राचीन पाण्डुलिपिहरूमा भेट्ने गरिन्छ। यो विभिन्न दक्षिणपूर्वी एसियाली लिपिहरूसँग पनि सम्बन्धित छ।
\end{document}