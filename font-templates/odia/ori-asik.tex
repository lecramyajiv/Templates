%
% Copyright 2023 Lecram Yajiv

% Licensed under the Apache License, Version 2.0 (the "License");
% you may not use this file except in compliance with the License.
% You may obtain a copy of the License at

% http://www.apache.org/licenses/LICENSE-2.0

% Unless required by applicable law or agreed to in writing, software
% distributed under the License is distributed on an "AS IS" BASIS,
% WITHOUT WARRANTIES OR CONDITIONS OF ANY KIND, either express or implied.
% See the License for the specific language governing permissions and
% limitations under the License.

% translated from wikipedia about chera dynasty
% % https://en-m-wikipedia-org.translate.goog/wiki/Chera_dynasty?_x_tr_sl=auto&_x_tr_tl=ml&_x_tr_hl=en-GB

\RequirePackage[orthodox]{nag}
\documentclass[a4paper,12pt,oneside,final]{article}
\usepackage{microtype}
\usepackage[a4paper, margin=1.3cm, nohead, nofoot]{geometry}
\usepackage{fontspec}
%\usepackage[babelshorthands=true]{polyglossia}
\usepackage{setspace}
%\setmainlanguage{bengali}
\setmainfont[Script=Oriya,Renderer=Harfbuzz]{Alkatra}
\pagestyle{empty}
\onehalfspacing
\begin{document}
ଓଡ଼ିଆ ଭାଷା ଓଡ଼ିଶାର ପାଖାପାଖି ଅନେକ ରାଜ୍ୟରେ କୁହାଯାଏ । ପଶ୍ଚିମବଙ୍ଗର ମେଦିନୀପୁର ସମେତ ଆହୁରି ଅନେକ ଅଞ୍ଚଳ, ଝାଡ଼ଖଣ୍ଡର ସିଂହଭୂମି ଜିଲ୍ଲା, ଆନ୍ଧ୍ର ପ୍ରଦେଶର ଶ୍ରୀକାକୁଲମ, ବିଜୟନଗର ଓ ବିଶାଖାପଟନମ ଜିଲ୍ଲା, ଛତିଶଗଡ଼ର ପୂର୍ବ ଜିଲ୍ଲା ଓ ନାଗପୁର ଅଞ୍ଚଳରେ ବହୁଳ ଭାବରେ କୁହାଯାଏ । ଗୁଜରାଟର ସୁରଟ ଭାରତର ଦ୍ୱିତୀୟ ଓଡ଼ିଆ ଭାଷା ବହୁଳ ଅଞ୍ଚଳ । ଏହାଛଡା ବେଙ୍ଗାଳୁରୁ, ହାଇଦ୍ରାବାଦ, ପଣ୍ଡିଚେରୀ, ଚେନ୍ନାଇ, ଗୋଆ, ମୁମ୍ବାଇ, ଜାମସେଦପୁର, ବରୋଦା, ରାୟପୁର, ଅହମଦାବାଦ, ଦିଲ୍ଲୀ, କଲିକତା, ଖଡ଼ଗପୁର, ଗୁଆହାଟୀ, ପୁନେ ଓ ସିଲଭାସା ଆଦି ସହରରେ ଓଡ଼ିଆ ଭାଷାଭାଷୀ ଲୋକ ଦେଖାଯାନ୍ତି ।
\end{document}