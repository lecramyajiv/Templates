%
% Copyright 2023 Lecram Yajiv

% Licensed under the Apache License, Version 2.0 (the "License");
% you may not use this file except in compliance with the License.
% You may obtain a copy of the License at

% http://www.apache.org/licenses/LICENSE-2.0

% Unless required by applicable law or agreed to in writing, software
% distributed under the License is distributed on an "AS IS" BASIS,
% WITHOUT WARRANTIES OR CONDITIONS OF ANY KIND, either express or implied.
% See the License for the specific language governing permissions and
% limitations under the License.
% translated from wikipedia about cheras
% % https://en-m-wikipedia-org.translate.goog/wiki/Chera_dynasty?_x_tr_sl=auto&_x_tr_tl=ml&_x_tr_hl=en-GB


\RequirePackage[orthodox]{nag}
\documentclass[a4paper,12pt,oneside,final]{article}
\usepackage{microtype}
\usepackage[a4paper, margin=1.3cm, nohead, nofoot]{geometry}
\usepackage[T1]{fontenc}
\usepackage{fontspec}
\usepackage{setspace}
\pagestyle{empty}
\onehalfspacing
\usepackage[marathi,provide=*]{babel}
\babelfont[marathi]{rm}[Renderer=Harfbuzz]{Poppins}
\begin{document}
मराठी, संस्कृत, हिंदी, कोंकणी, गुजराती, काश्मिरी, सिंधी, नेपाळी आणि रोमानीसारख्या या व इतर काही भारतीय मुळे असलेल्या भाषांची प्रथम लिपी ही देवनागरी आहे. देवनागरी लिपी ही अबुगिडा लेखनपद्धतीमध्ये मोडते. जगातल्या बहुसंख्य लिपींप्रमणे देवनागरीदेखील डावीकडून उजवीकडे लिहिली जाते. देवनागरी लिपीचा विकसनकाल हा बराच मोठा असून ह्या लिपीचा ख्रिस्तपूर्व ५०० मध्ये असलेल्या ब्राह्मी लिपीपासून आताची देवनागरी लिपीचा विकास झालेला आहे. साधारणत: इसवी सनाच्या बाराव्या शतकात स्थिरावलेल्या लेखनपद्धतीस देवनागरी असे नांव उपयोजिण्यास आरंभ झाला असावा. प्रत्येक शब्दावर एक रेषा ओढली जाते. तिला शिरोरेषा म्हणतात. तिच्यामुळे लेखन नीटनेटके व सुंदर दिसते. ही एक ध्वन्यात्मक लिपी आहे व त्यामुळे इतर लिप्यांपेक्षा (रोमन, अरबी, चिनी इत्यादी) अधिक वैज्ञानिक आहे. मराठी देवनागरी लिपीला बालबोध लिपी म्हणतात. (बालबोध नसलेली दुसरी मराठी लिपी म्हणजे मोडी लिपी.)
मराठी लेखक व कवी दासोपंत यांचे सुमारे १५व्या शतकातील लिहिलेले मराठी.

जगातील विविध भाषांतील बहुतांश शब्द किंवा ध्वनी देवनागरी लिपीमध्ये जवळजवळ जसेच्या तसे लिहिता येऊ शकतात आणि रोमन किंवा इतर लिप्यांपेक्षा देवनागरीत सहज लिहिलेल्या शब्दांचा तुलनात्मकदृष्ट्या हुबेहूब उच्चार करता येतो.

या लिपीत एकूण ५२ अक्षरे आहेत, ज्यात १६ स्वर आणि ३६ व्यंजने आहेत. अक्षरांचा क्रमसुद्धा वैज्ञानिक आहे. स्वर-व्यंजन, कोमल-कठोर, अल्पप्राण-महाप्राण, अनुनासिक-अन्तस्थ-उष्म इत्यादी वर्गीकरणही वैज्ञानिक आहे.

भारत तसेच आशिया मधील अनेक लिप्यांचे (उर्दू सोडून) संकेत देवनागरीपेक्षा वेगळे आहेत. पण उच्चारण व वर्ण-क्रम इत्यादी देवनागरीसारखेच आहेत. त्यामुळे लिप्यंतरण करणे सोपे जाते. देवनागरी लेखनाच्या दृष्टीने सरळ, सुंदर आणि वाचनाच्या दृष्टीने सुपाठ्य आहे.

आपले विचार शब्दांच्या किंवा कोणत्याही माध्यमातुन व्यक्त करण्याचे कार्य भाषा करते. तत्त्व: कुठलीही भाषा कोणत्याही लिपीत लिहिता येते. प्रत्यक्षात हे बरेचसे जमले तरी पूर्णपणे शक्य होत नाही. भाषेची लिपी त्या त्या भाषेतून उच्चारलेल्या शब्दांच्या लिखाणासाठी असते. एखाद्या भाषेत जर विशिष्ट उच्चार नसतील तर तिच्या लिपीतही ते दाखवणाऱ्या अक्षरखुणा नसतात. इंग्रजीत ख, च, छ, ठ, फ, घ, ढ, भ, ष, ळ हे उच्चार नाहीत. तमिळमध्ये ख, ग, घ, छ, ज, झ, ठ, ड, ढ, थ, द, ध, फ, ब, भ, श, स, ह ही अक्षरे लिहिता येत नाहीत, पण यांच्यापैकी काही उच्चार आहेत. त्यामुळे लिप्यंतरावर मर्यादा पडतात. उदाहरणार्थ मराठीतला 'पाटील" हा शब्द रोमन लिपीत Patil असा लिहिला जातो. त्याचा उच्चार पतिल/पतिळ/पॅतिल/पॅतिळ/पातिल/पातिळ/पाटिळ असा काहीही होऊ शकतो. देवनागरी लिपीत जगातल्या बहुसंख्य भाषांचे बहुतेक उच्चार जवळजवळ अचूक लिहिण्याची क्षमता असल्याने, तुलनात्मक दृष्ट्या या दृष्टिकोणातून देवनागरी ही एक उत्कृष्ट लिपी समजली जाते.
\end{document}