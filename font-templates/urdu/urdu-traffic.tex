%
% Copyright 2023 Lecram Yajiv

% Licensed under the Apache License, Version 2.0 (the "License");
% you may not use this file except in compliance with the License.
% You may obtain a copy of the License at

% http://www.apache.org/licenses/LICENSE-2.0

% Unless required by applicable law or agreed to in writing, software
% distributed under the License is distributed on an "AS IS" BASIS,
% WITHOUT WARRANTIES OR CONDITIONS OF ANY KIND, either express or implied.
% See the License for the specific language governing permissions and
% limitations under the License.
% translated from wikipedia about cheras
% % https://en-m-wikipedia-org.translate.goog/wiki/Chera_dynasty?_x_tr_sl=auto&_x_tr_tl=ml&_x_tr_hl=en-GB


\RequirePackage[orthodox]{nag}
\documentclass[a4paper,12pt,oneside,final]{article}
\usepackage{microtype}
\usepackage[a4paper, margin=1.3cm, nohead, nofoot]{geometry}
\usepackage[T1]{fontenc}
\usepackage{fontspec}
\usepackage{setspace}
\pagestyle{empty}
\onehalfspacing
\usepackage[urdu, bidi=basic-r, provide=*]{babel}
\babelfont[urdu]{rm}[Renderer=Harfbuzz]{XM Traffic}
\begin{document}
اُردُو، برصغیر پاک و ہند کی معیاری زبانوں میں سے ایک ہے۔ یہ پاکستان کی قومی اور رابطہ عامہ کی زبان ہے، جبکہ بھارت کی چھ ریاستوں کی دفتری زبان کا درجہ رکھتی ہے۔ آئین ہند کے مطابق اسے 22 دفتری شناخت شدہ زبانوں میں شامل کیا جا چکا ہے۔ 2001ء کی مردم شماری کے مطابق اردو کو بطور مادری زبان بھارت میں 5.01 فیصد لوگ بولتے ہیں اور اس لحاظ سے ہی بھارت کی چھٹی بڑی زبان ہے جبکہ پاکستان میں اسے بطور مادری زبان 7.59 فیصد لوگ استعمال کرتے ہیں، یہ پاکستان کی پانچویں بڑی زبان ہے۔ اردو تاریخی طور پر ہندوستان کی مسلم آبادی سے جڑی ہے۔[8] زبانِ اردو کو پہچان و ترقی اس وقت ملی جب برطانوی دور میں انگریز حکمرانوں نے اسے فارسی کی بجائے انگریزی کے ساتھ شمالی ہندوستان کے علاقوں اور جموں و کشمیر میں اسے 1846ء اور پنجاب میں 1849ء میں بطور دفتری زبان نافذ کیا۔ اس کے علاوہ خلیجی، یورپی، ایشیائی اور امریکی علاقوں میں اردو بولنے والوں کی ایک بڑی تعداد آباد ہے جو بنیادی طور پر جنوبی ایشیاء سے کوچ کرنے والے اہلِ اردو ہیں۔ 1999ء کے اعداد و شمار کے مطابق اردو زبان کے مجموعی متکلمین کی تعداد دس کروڑ ساٹھ لاکھ کے لگ بھگ تھی۔ اس لحاظ سے یہ دنیا کی نویں بڑی زبان ہے۔ اردو زبان کو کئی ہندوستانی ریاستوں میں سرکاری حیثیت بھی حاصل ہے۔ [9] نیپال میں، اردو ایک رجسٹرڈ علاقائی بولی ہے [10] اور جنوبی افریقہ میں یہ آئین میں ایک محفوظ زبان ہے۔ یہ افغانستان اور بنگلہ دیش میں اقلیتی زبان کے طور پر بھی بولی جاتی ہے، جس کی کوئی سرکاری حیثیت نہیں ہے۔

1837ء میں، اردو برطانوی ایسٹ انڈیا کمپنی کی سرکاری زبان بن گئی، کمپنی کے دور میں پورے شمالی ہندوستان میں فارسی کی جگہ لی گئی۔ فارسی اس وقت تک مختلف ہند-اسلامی سلطنتوں کی درباری زبان کے طور پر کام کرتی تھی۔ [11] یورپی نوآبادیاتی دور میں مذہبی، سماجی اور سیاسی عوامل پیدا ہوئے جنھوں نے اردو اور ہندی کے درمیان فرق کیا، جس کی وجہ سے ہندی-اردو تنازعہ شروع ہوا ۔
\end{document}