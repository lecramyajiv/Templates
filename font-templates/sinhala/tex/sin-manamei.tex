%
% Copyright 2023 Lecram Yajiv

% Licensed under the Apache License, Version 2.0 (the "License");
% you may not use this file except in compliance with the License.
% You may obtain a copy of the License at

% http://www.apache.org/licenses/LICENSE-2.0

% Unless required by applicable law or agreed to in writing, software
% distributed under the License is distributed on an "AS IS" BASIS,
% WITHOUT WARRANTIES OR CONDITIONS OF ANY KIND, either express or implied.
% See the License for the specific language governing permissions and
% limitations under the License.
% translated from wikipedia about cheras
% % https://en-m-wikipedia-org.translate.goog/wiki/Chera_dynasty?_x_tr_sl=auto&_x_tr_tl=ml&_x_tr_hl=en-GB


\RequirePackage[orthodox]{nag}
\documentclass[a4paper,12pt,oneside,final]{article}
\usepackage{microtype}
\usepackage[a4paper, margin=1.3cm, nohead, nofoot]{geometry}
\usepackage[T1]{fontenc}
\usepackage{fontspec}
\usepackage{setspace}
\pagestyle{empty}
\onehalfspacing
\usepackage[sinhala, provide=*]{babel}
\babelfont[sinhala]{rm}[Renderer=Harfbuzz]{ManameInformal}
\begin{document}
සිංහල භාෂාව ලිවීමට යොදාගන්නා අකුරු සිංහල අකුරු නම්වේ. සිංහල බ්‍රාහ්මීය අක්ෂර අබුගිඩා වලින් පැවත එන නමුත්, වෙනත් බ්‍රාහ්මීය අක්ෂර සමග සසඳන විට විශාල වෙනස්කම් දක්වයි. මෙම හේතුව නිසා සිංහල යුනිකෝඩ් පිටුව අනෙක් අක්ෂර වල යුනිකෝඩ් පිටු වලට වෙනස් වෙයි

පාලි සහ සංස්කෘත භාෂා සිංහල භාෂාවට බලපෑ ප්‍රබල මව් භාෂා දෙකකි. මෙම මවු භාෂාවල වචන එදිනෙද සිංහල භාෂක පිරිස අතර භාවිත වන අතර එම වචන සිංහල අයුරින් සිංහල අක්ෂර භාවිත කොට සන්නිවේදනයේදී භාවිත කෙරෙයි. එසේම සිංහල විද්‍යාර්ථයින් පාලි භාෂාවටම වෙනම අක්ෂර ක්‍රමයක් නොමැති නිසා පාලි භාෂාව ඉගැනීමේදී ඒවා පාලි වචන සිංහල අකුරින් ලියන අතර සංස්කෘත භාෂාව ඉගෙනීමේ පහසුව හෝ දේවනගරි අක්ෂර නොදැනුවත්කම නිසා නැතහොත් සන්නිවේදන පහසුව සඳහාද සිංහල භාෂක පිරිස් සංස්කෘත භාෂාවේ වචන සිංහල අකුරින් ලියති.

මෙම ගද්‍ය ඛණ්ඩය අනුව, ලුහුගුරු (කෙටි දීර්ඝ) ප්‍රාණාක්‍ෂර 10 ක් ද, ගාත්‍රාක්‍ෂර 20 ක් ද සහිත වූ සිදත් සඟරා හෝඩියෙහි අක්‍ෂර මාලාව 30 කි. (පහත "සිදත් සඟරා හෝඩිය" බලන්න.)

මෙම හෝඩියේ ‘ඇ, ඈ සහ අඃ’ නැත. මෙහි ඉහත අක්‍ෂර අන්තර්ගත නොවූව ද සාමාන්‍ය ව්‍යවහාරයේ පැවති බැවින් විවිධ තර්ක විතර්ක පැන නැගුණි. ඒ මත ‘ඇ, ඈ’ අක්‍ෂර දෙක සහිතව ශුද්ධ සිංහල අක්‍ෂර මාලාව නිර්මාණය විය. මින් අනතුරුව සංස්කෘත භාෂාවේ බලපෑම මත ප්‍රාථමික විස්තෘත සිංහල හෝඩිය නිර්මාණය වේ. වදන් කවි පොතෙහි අන්තර්ගත මෙහි සම්පූර්ණ අක්‍ෂර 50 කි. ස්වර 14 ක් හා ව්‍යංජන 36 ක් එහි ඇත. ඉන් අනනතුරුව නිර්මාණය වූ විස්තෘත සිංහල හෝඩිය (මිශ්‍ර සිංහල හෝඩිය) අක්‍ෂර 54 න් යුක්ත වේ. මෙහි අන්තර්ගත “ඇ, ඈ, ඒ, ඕ” අක්‍ෂර 04 මිශ්‍ර සිංහල හෝඩියේ අන්තර් ගත නොවේ.

ලේඛණ වියවුල් හා විවිධ ගැටළු පැනනැගීම හේතුවෙන් 80 දශකයේ අගභාගයේ ජාතික අධ්ආ‍යාපණ ආයතනය මගින් සිංහල ලේඛණ රීතිය නමින් පොතක් එලි දක්‍වමින්, නූතන සිංහල අක්‍ෂර මාලාව හදුන්වා දුනි. මෙහි සම්පූර්ණ අක්‍ෂර 60 කි. මිශ්‍ර සිංහල හෝඩියට සඤ්ඤක අක්‍ෂර 5ක් සහ "ෆ" අක්‍ෂරය එකතු වීමෙන් නූතන සිංහල අක්‍ෂර මාලාව නිර්මාණය වී ඇත. මෙහි ස්වර 18 ක් හා ව්‍යංජන 42 ක් අන්තර්ගත වේ.
\end{document}