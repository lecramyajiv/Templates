%
% Copyright 2023 Lecram Yajiv

% Licensed under the Apache License, Version 2.0 (the "License");
% you may not use this file except in compliance with the License.
% You may obtain a copy of the License at

% http://www.apache.org/licenses/LICENSE-2.0

% Unless required by applicable law or agreed to in writing, software
% distributed under the License is distributed on an "AS IS" BASIS,
% WITHOUT WARRANTIES OR CONDITIONS OF ANY KIND, either express or implied.
% See the License for the specific language governing permissions and
% limitations under the License.
% translated from wikipedia about cheras
% %
% https://ar.wikipedia.org/wiki/%D8%A7%D9%84%D9%84%D8%BA%D8%A9_%D8%A7%D9%84%D8%B9%D8%B1%D8%A8%D9%8A%D8%A9

\RequirePackage[orthodox]{nag}
\documentclass[a4paper,12pt,oneside,final]{article}
\usepackage{microtype}
\usepackage[a4paper, margin=1.3cm, nohead, nofoot]{geometry}
\usepackage[T1]{fontenc}
\usepackage{fontspec}
\usepackage{setspace}
\pagestyle{empty}
\onehalfspacing
\usepackage[arabic, bidi=basic-r, provide=*]{babel}
\babelfont[arabic]{rm}[Renderer=Harfbuzz]{Ouhod}
\begin{document}
ٱللُّغَةُ ٱلْعَرَبِيَّة هي أكثر اللغات السامية تحدثًا، وإحدى أكثر اللغات انتشاراً في العالم، يتحدثها أكثر من 467 مليون نسمة.(1) ويتوزع متحدثوها في الوطن العربي، بالإضافة إلى العديد من المناطق الأخرى المجاورة كالأحواز وتركيا وتشاد ومالي والسنغال وإرتيريا وإثيوبيا وجنوب السودان وإيران. وبذلك فهي تحتل المركز الرابع أو الخامس من حيث اللغات الأكثر انتشارًا في العالم، وهي تحتل المركز الثالث تبعًا لعدد الدول التي تعترف بها كلغة رسمية؛ إذ تعترف بها 27 دولة لغةً رسميةً، واللغة الرابعة من حيث عدد المستخدمين على الإنترنت. اللغةُ العربيةُ ذات أهمية قصوى لدى المسلمين، فهي عندَهم لغةٌ مقدسة إذ أنها لغة القرآن، وهي لغةُ الصلاة وأساسيةٌ في القيام بالعديد من العبادات والشعائرِ الإسلامية. العربيةُ هي أيضاً لغة شعائرية رئيسية لدى عدد من الكنائس المسيحية في الوطن العربي، كما كُتبَت بها كثير من أهمِّ الأعمال الدينية والفكرية اليهودية في العصور الوسطى. ارتفعتْ مكانةُ اللغةِ العربية إثْرَ انتشارِ الإسلام بين الدول إذ أصبحت لغة السياسة والعلم والأدب لقرون طويلة في الأراضي التي حكمها المسلمون. وللغة العربية تأثير مباشر وغير مباشر على كثير من اللغات الأخرى في العالم الإسلامي، كالتركية والفارسية والأمازيغية والكردية والأردية والماليزية والإندونيسية والألبانية وبعض اللغات الإفريقية الأخرى مثل الهاوسا والسواحيلية والتجرية والأمهرية والصومالية، وبعض اللغات الأوروبية وخاصةً المتوسطية كالإسبانية والبرتغالية والمالطية والصقلية؛ ودخلت الكثير من مصطلحاتها في اللغة الإنجليزية واللغات الأخرى، مثل أدميرال والتعريفة والكحول والجبر وأسماء النجوم. كما أنها تُدرَّس بشكل رسمي أو غير رسمي في الدول الإسلامية والدول الإفريقية المحاذية للوطن العربي.
\end{document}