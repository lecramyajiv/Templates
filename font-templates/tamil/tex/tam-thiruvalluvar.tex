%
% Copyright 2023 Lecram Yajiv

% Licensed under the Apache License, Version 2.0 (the "License");
% you may not use this file except in compliance with the License.
% You may obtain a copy of the License at

% http://www.apache.org/licenses/LICENSE-2.0

% Unless required by applicable law or agreed to in writing, software
% distributed under the License is distributed on an "AS IS" BASIS,
% WITHOUT WARRANTIES OR CONDITIONS OF ANY KIND, either express or implied.
% See the License for the specific language governing permissions and
% limitations under the License.

% translated from wikipedia about chera dynasty
% % https://en-m-wikipedia-org.translate.goog/wiki/Chera_dynasty?_x_tr_sl=auto&_x_tr_tl=ml&_x_tr_hl=en-GB

\RequirePackage[orthodox]{nag}
\documentclass[a4paper,12pt,oneside,final]{article}
\usepackage{microtype}
\usepackage[a4paper, margin=1.3cm, nohead, nofoot]{geometry}
\usepackage{fontspec}
%\usepackage[babelshorthands=true]{polyglossia}
\usepackage{setspace}
%\setmainlanguage{bengali}
\setmainfont[Script=Tamil]{ThiruValluvar-Regular}
\pagestyle{empty}
\onehalfspacing
\begin{document}
சடையவர்மன் பராக்கிரம பாண்டியன் முதல் அவனின் அடுத்த வந்த பாண்டியர் அனைவரும் தென்காசிப் பாண்டியர்கள் எனப்படுவர்.[1] பதினான்காம் நூற்றாண்டு முதல் தமிழகத்தில் ஏற்பட்ட சுல்தானியர், விஜயநகரத்தவர், நாயக்கர் படையெடுப்புகளால் பாண்டியர் தங்கள் பாரம்பரியத் தலைநகரான மதுரையை இழந்து தென்காசி, திருநெல்வேலி போன்ற தென்தமிழக நகரங்களில் சிற்றரசர்களாக வாழத் தலைப்பட்டனர். பாண்டியர்களின் கடைசித் தலைநகரம் தென்காசி ஆகும்.[2][3] சடையவர்மன் பராக்கிரம பாண்டியன் முதல் அவனின் அடுத்த தலைமுறையில் வந்த அனைத்து பாண்டியரும் தென்காசியையே தலைநகராகக் கொண்டு தென்காசி பெரியகோயிலில்[4] உள்ள சிவந்தபாதவூருடைய ஆதீன மடத்தில்[5] முடி சூட்டிக்கொண்டனர். அதே காலத்தில் சில பாண்டியர் நெல்லையையும் தலைநகரமாகக் கொண்டு ஆண்டு வந்தனர். கயத்தார், வள்ளியூர், உக்கிரன் கோட்டை போன்ற நகரங்களும் இவர்களின் முக்கிய நகரங்களாகும். தென்காசி பெரியகோயில், பிரம்மதேசம், சேரன்மாதேவி, அம்பாசமுத்திரம், களக்காடு, புதுக்கோட்டை ஆகிய ஊர்களில் இவர்களைப் பற்றிய கல்வெட்டுகள் மற்றும் செப்பேடுகளும் காணப்படுகின்றன. தென்காசிப் பாண்டியர்களில் கொல்லங்கொண்டான் என்பவனே பாண்டியர் வரலாற்றில் அறியப்படும் கடைசி பாண்டிய மன்னனாவான்.
\end{document}