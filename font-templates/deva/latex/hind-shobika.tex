%
% Copyright 2023 Lecram Yajiv

% Licensed under the Apache License, Version 2.0 (the "License");
% you may not use this file except in compliance with the License.
% You may obtain a copy of the License at

% http://www.apache.org/licenses/LICENSE-2.0

% Unless required by applicable law or agreed to in writing, software
% distributed under the License is distributed on an "AS IS" BASIS,
% WITHOUT WARRANTIES OR CONDITIONS OF ANY KIND, either express or implied.
% See the License for the specific language governing permissions and
% limitations under the License.
% translated from wikipedia about cheras
% % https://en-m-wikipedia-org.translate.goog/wiki/Chera_dynasty?_x_tr_sl=auto&_x_tr_tl=ml&_x_tr_hl=en-GB


\RequirePackage[orthodox]{nag}
\documentclass[a4paper,12pt,oneside,final]{article}
\usepackage{microtype}
\usepackage[a4paper, margin=1.3cm, nohead, nofoot]{geometry}
\usepackage[T1]{fontenc}
\usepackage{fontspec}
\usepackage{setspace}
\pagestyle{empty}
\onehalfspacing
\usepackage[hindi,provide=*]{babel}
\babelfont[hindi]{rm}[Renderer=Harfbuzz]{Shobhika}
\begin{document}
देवनागरी भारतीय उपमहाद्वीप में प्रयुक्त प्राचीन ब्राह्मी लिपि पर आधारित बाएँ से दाएँ आबूगीदा है। यह प्राचीन भारत में पहली से चौथी शताब्दी ईस्वी तक विकसित किया गया था और ७वीं शताब्दी ईस्वी तक नियमित उपयोग में था। देवनागरी लिपि, जिसमें १४ स्वर और ३३ व्यञ्जन सहित ४७ प्राथमिक वर्ण हैं, दुनिया में चौथी सबसे व्यापक रूप से अपनाई जाने वाली लेखन प्रणाली है, जिसका उपयोग १२० से अधिक भाषाओं के लिए किया जा रहा है। [4]

इस लिपि की शब्दावली भाषा के उच्चारण को दर्शाती है। रोमन लिपि के विपरीत, इस लिपि में अक्षर केस की कोई अवधारणा नहीं है। यह बाएँ से दाएँ लिखा गया है, चौकोर रूपरेखा के भीतर सममित गोल आकृतियों के लिए एक दृढ़ प्राथमिकता है, और एक क्षैतिज रेखा द्वारा पहचाना जा सकता है, जिसे शिरोरेखा के रूप में जाना जाता है, जो पूर्ण अक्षरों के शीर्ष के साथ चलती है। एक सरसरी दृष्टि में, देवनागरी लिपि अन्य भारतीय लिपियों जैसे पूर्वी नागरी लिपि या गुरमुखी लिपि से अलग दिखाई देती है, लेकिन एक निकटतम अवलोकन से पता चलता है कि वे कोण और संरचनात्मक जोर को छोड़कर बहुत समान हैं।देवनागरी भारतीय उपमहाद्वीप में प्रयुक्त प्राचीन ब्राह्मी लिपि पर आधारित बाएँ से दाएँ आबूगीदा है। यह प्राचीन भारत में पहली से चौथी शताब्दी ईस्वी तक विकसित किया गया था और ७वीं शताब्दी ईस्वी तक नियमित उपयोग में था। देवनागरी लिपि, जिसमें १४ स्वर और ३३ व्यञ्जन सहित ४७ प्राथमिक वर्ण हैं, दुनिया में चौथी सबसे व्यापक रूप से अपनाई जाने वाली लेखन प्रणाली है, जिसका उपयोग १२० से अधिक भाषाओं के लिए किया जा रहा है। [4]

इस लिपि की शब्दावली भाषा के उच्चारण को दर्शाती है। रोमन लिपि के विपरीत, इस लिपि में अक्षर केस की कोई अवधारणा नहीं है। यह बाएँ से दाएँ लिखा गया है, चौकोर रूपरेखा के भीतर सममित गोल आकृतियों के लिए एक दृढ़ प्राथमिकता है, और एक क्षैतिज रेखा द्वारा पहचाना जा सकता है, जिसे शिरोरेखा के रूप में जाना जाता है, जो पूर्ण अक्षरों के शीर्ष के साथ चलती है। एक सरसरी दृष्टि में, देवनागरी लिपि अन्य भारतीय लिपियों जैसे पूर्वी नागरी लिपि या गुरमुखी लिपि से अलग दिखाई देती है, लेकिन एक निकटतम अवलोकन से पता चलता है कि वे कोण और संरचनात्मक जोर को छोड़कर बहुत समान हैं।

अधिकतर भाषाओं की तरह देवनागरी भी बायें से दायें लिखी जाती है। प्रत्येक शब्द के ऊपर एक रेखा खिंची होती है (कुछ वर्णों के ऊपर रेखा नहीं होती है) जिसे शिरोरेखा कहते हैं। देवनागरी का विकास ब्राह्मी लिपि से हुआ है। यह एक ध्वन्यात्मक लिपि है जो प्रचलित लिपियों (रोमन, अरबी, चीनी आदि) में सबसे अधिक वैज्ञानिक है। इससे वैज्ञानिक और व्यापक लिपि शायद केवल अध्वव लिपि है। भारत की कई लिपियाँ देवनागरी से बहुत अधिक मिलती-जुलती हैं, जैसे- बांग्ला, गुजराती, गुरुमुखी आदि। कम्प्यूटर प्रोग्रामों की सहायता से भारतीय लिपियों को परस्पर परिवर्तन बहुत आसान हो गया है।

भारतीय भाषाओं के किसी भी शब्द या ध्वनि को देवनागरी लिपि में ज्यों का त्यों लिखा जा सकता है और फिर लिखे पाठ को लगभग 'हू-ब-हू' उच्चारण किया जा सकता है, जो कि रोमन लिपि और अन्य कई लिपियों में सम्भव नहीं है, जब तक कि उनका विशेष मानकीकरण न किया जाये, जैसे आइट्रांस या IAST ।

इसमें कुल ५२ अक्षर हैं, जिसमें १४ स्वर और ३८ व्यंजन हैं। अक्षरों की क्रम व्यवस्था (विन्यास) भी बहुत ही वैज्ञानिक है। स्वर-व्यंजन, कोमल-कठोर, अल्पप्राण-महाप्राण, अनुनासिक्य-अन्तस्थ-उष्म इत्यादि वर्गीकरण भी वैज्ञानिक हैं। एक मत के अनुसार देवनगर (काशी) में प्रचलन के कारण इसका नाम देवनागरी पड़ा।

भारत तथा एशिया की अनेक लिपियों के संकेत देवनागरी से अलग हैं, परन्तु उच्चारण व वर्ण-क्रम आदि देवनागरी के ही समान हैं, क्योंकि वे सभी ब्राह्मी लिपि से उत्पन्न हुई हैं (उर्दू को छोड़कर)। इसलिए इन लिपियों को परस्पर आसानी से लिप्यन्तरित किया जा सकता है। देवनागरी लेखन की दृष्टि से सरल, सौन्दर्य की दृष्टि से सुन्दर और वाचन की दृष्टि से सुपाठ्य है।
\end{document}