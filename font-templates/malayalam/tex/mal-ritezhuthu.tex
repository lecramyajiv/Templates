%
% Copyright 2023 Lecram Yajiv

% Licensed under the Apache License, Version 2.0 (the "License");
% you may not use this file except in compliance with the License.
% You may obtain a copy of the License at

% http://www.apache.org/licenses/LICENSE-2.0

% Unless required by applicable law or agreed to in writing, software
% distributed under the License is distributed on an "AS IS" BASIS,
% WITHOUT WARRANTIES OR CONDITIONS OF ANY KIND, either express or implied.
% See the License for the specific language governing permissions and
% limitations under the License.
% translated from wikipedia about cheras
% % https://en-m-wikipedia-org.translate.goog/wiki/Chera_dynasty?_x_tr_sl=auto&_x_tr_tl=ml&_x_tr_hl=en-GB


\RequirePackage[orthodox]{nag}
\documentclass[a4paper,12pt,oneside,final]{article}
\usepackage{microtype}
\usepackage[a4paper, margin=1.3cm, nohead, nofoot]{geometry}
\usepackage{fontspec}
\usepackage{setspace}
\pagestyle{empty}
\onehalfspacing
\usepackage[malayalam, provide=*]{babel}
\babelfont[malayalam]{rm}[Renderer=Harfbuzz]{RIT-Ezhuthu-Regular}
\begin{document}
ചേര രാജവംശം , ഒരു സംഘകാല രാജവംശമായിരുന്നു ദക്ഷിണേന്ത്യയിലെ പടിഞ്ഞാറൻ തീരത്തിന്റെയും പശ്ചിമഘട്ടത്തിന്റെയും വിവിധ പ്രദേശങ്ങൾ ഏകീകരിച്ച് ആദ്യകാല ചേര സാമ്രാജ്യം രൂപീകരിച്ചു.വിസ്തൃതമായ ഇന്ത്യൻ മഹാസമുദ്ര ശൃംഖലകൾ വഴിയുള്ള സമുദ്രവ്യാപാരത്തിൽ നിന്ന് ലാഭം നേടാൻ ചേര രാജ്യം ഭൂമിശാസ്ത്രപരമായി മികച്ച സ്ഥാനത്തായിരുന്നു. മിഡിൽ ഈസ്റ്റേൺ, ഗ്രേക്കോ-റോമൻ വ്യാപാരികളുമായുള്ള സുഗന്ധവ്യഞ്ജനങ്ങൾ , പ്രത്യേകിച്ച് കുരുമുളക് കൈമാറ്റം പല സ്രോതസ്സുകളിൽ സാക്ഷ്യപ്പെടുത്തിയിട്ടുണ്ട്.ആദ്യകാല ചരിത്ര കാലഘട്ടത്തിലെ (സി. ബി.സി. രണ്ടാം നൂറ്റാണ്ട് - സി. സി. മൂന്നാം നൂറ്റാണ്ട് [6] ) ചേരന്മാരുടെ യഥാർത്ഥ കേന്ദ്രം കേരളത്തിലെ കുട്ടനാട്ടിലും തുറമുഖങ്ങൾ മുച്ചിരിയിലും ഉണ്ടായിരുന്നു . മുസിരിസ്) , തൊണ്ടി (ടിണ്ടിസ്) എന്നിവ ഇന്ത്യൻ മഹാസമുദ്ര തീരത്തും ( കേരളം ) കൊങ്ങുനാടും . മലബാർ തീരത്തിന്റെ തെക്ക് ആലപ്പുഴ മുതൽ വടക്ക് കാസർഗോഡ് വരെയുള്ള പ്രദേശം അവർ ഭരിച്ചു . കോയമ്പത്തൂരിന് ചുറ്റുമുള്ള പ്രദേശം സംഘകാലഘട്ടത്തിൽ ചേരന്മാരാണ് ഭരിച്ചിരുന്നത് .  1- ഉം 4-ഉം നൂറ്റാണ്ടുകളിൽ മലബാർ തീരത്തിനും തമിഴ്‌നാടിനും ഇടയിലുള്ള പ്രധാന വ്യാപാര
പാതയായ പാലക്കാട് വിടവിലേക്കുള്ള കിഴക്കൻ കവാടമായി ഇത് പ്രവർത്തിച്ചു . [7] എന്നിരുന്നാലും ഇന്നത്തെ കേരള സംസ്ഥാനത്തിന്റെ തെക്കൻ
പ്രദേശം ( തിരുവനന്തപുരത്തിനും തെക്കൻ ആലപ്പുഴയ്ക്കും ഇടയിലുള്ള തീരപ്രദേശം) മധുരയിലെ പാണ്ഡ്യ രാജവംശവുമായി കൂടുതൽ ബന്ധപ്പെട്ടിരുന്ന ആയ്
രാജവംശത്തിന്റെ കീഴിലായിരുന്നു . [8]

ആദ്യകാല പല്ലവത്തിനു മുമ്പുള്ള [9] രാഷ്‌ട്രീയങ്ങളെ പലപ്പോഴും "കുടുംബ-അധിഷ്ഠിത പുനർവിതരണ സമ്പദ്‌വ്യവസ്ഥ" എന്ന് വിശേഷിപ്പിക്കാറുണ്ട്, പ്രധാനമായും "പാസ്റ്ററൽ-കം-കാർഷിക ഉപജീവനവും" "കൊള്ളയടിക്കുന്ന രാഷ്ട്രീയവും" രൂപപ്പെടുത്തിയതാണ്. [6] പഴയ തമിഴ് ബ്രാഹ്മി ഗുഹ ലേബൽ ലിഖിതങ്ങൾ, പെരും കടുങ്കോയുടെ മകൻ ഇളം കടുങ്കോയെയും ഇരുമ്പൊറൈ വംശത്തിലെ കോ അത്തൻ ചേരലിന്റെ ചെറുമകനെയും വിവരിക്കുന്നു. [10] [11] ബ്രാഹ്മി ഐതിഹ്യങ്ങളുള്ള ആലേഖനം ചെയ്ത പോർട്രെയിറ്റ് നാണയങ്ങൾ അനേകം ചേരനാമങ്ങൾ നൽകുന്നു, [12] ചേര ചിഹ്നങ്ങളായ വില്ലിന്റെയും അമ്പിന്റെയും പിൻഭാഗത്ത് ചിത്രീകരിച്ചിരിക്കുന്നു. [12] ആദ്യകാല സംഘഗ്രന്ഥങ്ങളുടെ സമാഹാരങ്ങളാണ് ആദ്യകാല ചേരന്മാരെക്കുറിച്ചുള്ള വിവരങ്ങളുടെ പ്രധാന ഉറവിടം. [3] സംഘത്തിന്റെ ഇതിഹാസ കാവ്യമായ ചിലപ്പതികാരത്തിലെ പ്രധാന സ്ത്രീ കഥാപാത്രമായ കണ്ണകിയെ ചുറ്റിപ്പറ്റിയുള്ള പാരമ്പര്യങ്ങൾക്ക് പേരുകേട്ടതാണ് ചെങ്കുട്ടുവൻ അല്ലെങ്കിൽ നല്ല ചേര . [4] [13] ആദ്യകാല ചരിത്ര കാലഘട്ടത്തിന്റെ അവസാനത്തിനുശേഷം, ഏകദേശം CE 3-5-ആം നൂറ്റാണ്ടിൽ, ചേരരുടെ ശക്തി ഗണ്യമായി കുറയുന്ന ഒരു കാലഘട്ടം ഉണ്ടായതായി തോന്നുന്നു. [14]

മധ്യകാലഘട്ടത്തിന്റെ തുടക്കത്തിൽ കിഴക്കൻ കേരളവും നിലവിലെ പടിഞ്ഞാറൻ തമിഴ്‌നാടിന്റെ ഏതാനും കിലോമീറ്ററുകൾ മാത്രമാണ് കൊങ്കുരാജ്യത്തിലെ ചേരന്മാർ നിയന്ത്രിച്ചിരുന്നതെന്ന് അറിയപ്പെടുന്നു . [15] ഇന്നത്തെ മദ്ധ്യകേരളവും കൊങ്കു ചേരരും ഏകദേശം 8-9 നൂറ്റാണ്ടിൽ വേർപിരിഞ്ഞ് ചേര പെരുമാൾ രാജ്യവും കൊങ്കു ചേര രാജ്യവും (സി. 9-12 നൂറ്റാണ്ട്) രൂപീകരിച്ചു. [16] ചേര ഭരണാധികാരികളുടെ വിവിധ ശാഖകൾ തമ്മിലുള്ള ബന്ധത്തിന്റെ കൃത്യമായ സ്വഭാവം വ്യക്തമല്ല. ഇതിനുശേഷം കേരളത്തിന്റെ ഇന്നത്തെ ഭാഗങ്ങളും കൊങ്ങുനാടും സ്വയംഭരണാവകാശം പ്രാപിച്ചു. [17] മധ്യകാല ദക്ഷിണേന്ത്യയിലെ ചില പ്രധാന രാജവംശങ്ങൾ - ചാലൂക്യ, പല്ലവ, പാണ്ഡ്യ, രാഷ്ട്രകൂട, ചോള - - കൊങ്കു ചേര രാജ്യം കീഴടക്കിയതായി തോന്നുന്നു. 10/11 നൂറ്റാണ്ടിൽ പാണ്ഡ്യ രാഷ്ട്രീയ വ്യവസ്ഥയിൽ കൊങ്കു ചേരന്മാർ ലയിച്ചതായി കാണുന്നു. പെരുമാൾ രാജ്യത്തിന്റെ പിരിച്ചുവിടലിനു ശേഷവും, രാജകീയ ലിഖിതങ്ങളും ക്ഷേത്ര ഗ്രാന്റുകളും, പ്രത്യേകിച്ച് കേരളത്തിന് പുറത്ത് നിന്ന്, രാജ്യത്തെയും ജനങ്ങളെയും "ചേരർ അല്ലെങ്കിൽ കേരളക്കാർ" എന്ന് വിളിക്കുന്നത് തുടർന്നു. [14]

തെക്കൻ കേരളത്തിലെ കൊല്ലം തുറമുഖത്തിന് പുറത്തുള്ള വേണാട് (വേണാട് ചേരന്മാർ അല്ലെങ്കിൽ "കുലശേഖരർ") ഭരണാധികാരികൾ പെരുമാളിൽ നിന്ന് തങ്ങളുടെ വംശപരമ്പര അവകാശപ്പെട്ടു. [14] [18] ഇന്നത്തെ മലപ്പുറം ജില്ലയിലെ തിരൂരങ്ങാടി , തിരൂർ താലൂക്കുകളുടെ ഭാഗങ്ങൾ ഉൾപ്പെട്ടിരുന്ന കോഴിക്കോട് സാമൂതിരിയുടെ സാമ്രാജ്യത്തിലെ ഒരു പഴയ പ്രവിശ്യയുടെ പേരും ചേരനാട് ആയിരുന്നു . [19] പിന്നീട് മലബാർ ബ്രിട്ടീഷ് ഭരണത്തിൻ കീഴിലായപ്പോൾ ഇത് മലബാർ ജില്ലയുടെ താലൂക്കായി മാറി . [19] [20] തിരൂരങ്ങാടി പട്ടണമായിരുന്നു ചേരനാട് താലൂക്കിന്റെ ആസ്ഥാനം . [19] [20] പിന്നീട് താലൂക്ക് ഏറനാട് താലൂക്കിൽ ലയിച്ചു . [19] [20]
\end{document}