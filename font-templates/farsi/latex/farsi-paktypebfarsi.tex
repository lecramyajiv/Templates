%
% Copyright 2023 Lecram Yajiv

% Licensed under the Apache License, Version 2.0 (the "License");
% you may not use this file except in compliance with the License.
% You may obtain a copy of the License at

% http://www.apache.org/licenses/LICENSE-2.0

% Unless required by applicable law or agreed to in writing, software
% distributed under the License is distributed on an "AS IS" BASIS,
% WITHOUT WARRANTIES OR CONDITIONS OF ANY KIND, either express or implied.
% See the License for the specific language governing permissions and
% limitations under the License.
% translated from wikipedia about cheras
% % https://en-m-wikipedia-org.translate.goog/wiki/Chera_dynasty?_x_tr_sl=auto&_x_tr_tl=ml&_x_tr_hl=en-GB


\RequirePackage[orthodox]{nag}
\documentclass[a4paper,12pt,oneside,final]{article}
\usepackage{microtype}
\usepackage[a4paper, margin=1.3cm, nohead, nofoot]{geometry}
\usepackage[T1]{fontenc}
\usepackage{fontspec}
\usepackage{setspace}
\pagestyle{empty}
\onehalfspacing
\usepackage[persian, bidi=basic-r, provide=*]{babel}
\babelfont[persian]{rm}[Renderer=Harfbuzz]{PakType Naskh Basic Farsi}
\begin{document}
 قارهٔ آفریقا قرار دارد. دهانهٔ باب‌المندب آن را به اقیانوس هند پیوند می‌دهد. کانال سوئز در شمال آن را به دریای مدیترانه وصل می‌کند. همسایگان این دریا کشورهای عربستان سعودی، یمن، جیبوتی، اریتره، سودان، مصر، اسرائیل و اردن می‌باشند. این دریا حد فاصل میان دو قارهٔ آسیا و آفریقا است.
مساحت آن ۴۳۸٬۰۰۰ کیلومتر مربع است که از این حیث پانزدهمین دریای جهان به‌شمار می‌رود. دریای سرخ حدود ۲٬۲۵۰ کیلومتر طول دارد و وسیع ترین قسمت آن ۳۵۵ کیلومتر پهنا دارد. عمیق‌ترین نقطهٔ آن حدود ۳۰۴۰ متر عمق دارد و عمق متوسط این دریا ۴۹۰ متر است.[۱] دریای سرخ از طریق کانال سوئز به دریای مدیترانه و از طریق باب‌المندب (تنگه مندب) با خلیج عدن و اقیانوس هند ارتباط دارد. همچنین خلیج ایلات در دریای سرخ واقع شده‌است.
پیشتر به این دریا، دریای قلزم گفته می‌شد. آن را بحر العربی (به عربی: بحر العربی)[۲] و دریای حجاز نیز می‌نامیدند.[۳] برخی بر این باورند که داستان گذشتن بنی‌اسرائیل و موسی از دریا که در تورات آمده به دریای سرخ اشاره دارد.[نیازمند منبع]
\end{document}