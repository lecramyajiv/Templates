%
% Copyright 2023 Lecram Yajiv

% Licensed under the Apache License, Version 2.0 (the "License");
% you may not use this file except in compliance with the License.
% You may obtain a copy of the License at

% http://www.apache.org/licenses/LICENSE-2.0

% Unless required by applicable law or agreed to in writing, software
% distributed under the License is distributed on an "AS IS" BASIS,
% WITHOUT WARRANTIES OR CONDITIONS OF ANY KIND, either express or implied.
% See the License for the specific language governing permissions and
% limitations under the License.

\RequirePackage[orthodox]{nag}
\documentclass[a4paper,12pt,oneside,final]{article}
\usepackage{microtype}
\usepackage[a4paper, margin=1.3cm, nohead, nofoot]{geometry}
\usepackage{fontspec}
\setmainfont[Script=Bengali]{Atma}
\begin{document}
একটি নদী হল একটি প্রাকৃতিক প্রবাহিত জলধারা , সাধারণত একটি মিষ্টি জলের স্রোত , যা পৃথিবীর স্থলভাগে বা গুহার অভ্যন্তরে নিম্ন উচ্চতায় অন্য জলাশয়ের দিকে প্রবাহিত হয় , যেমন একটি মহাসাগর , সমুদ্র , উপসাগর , হ্রদ , জলাভূমি বা অন্য নদী। কিছু ক্ষেত্রে, একটি নদী ভূমিতে প্রবাহিত হয় বা তার গতিপথের শেষে শুকিয়ে যায় অন্য জলের অংশে না পৌঁছে। ছোট নদীগুলিকে ক্রিক , ব্রুক এবং রিভুলেটের মতো নামে উল্লেখ করা যেতে পারে । ভৌগলিক বৈশিষ্ট্যের জন্য প্রযোজ্য সাধারণ শব্দ নদীর জন্য কোন সরকারী সংজ্ঞা নেই ,  যদিও কিছু দেশ বা সম্প্রদায়ে, একটি স্রোত তার আকার দ্বারা সংজ্ঞায়িত করা হয়। ছোট নদীর অনেক নাম ভৌগলিক অবস্থানের জন্য নির্দিষ্ট; উদাহরণ হল মার্কিন যুক্তরাষ্ট্রের কিছু অংশে "রান" , স্কটল্যান্ড এবং উত্তরপূর্ব ইংল্যান্ডে " বার্ন " , এবং উত্তর ইংল্যান্ডে "বেক" । কখনও কখনও একটি নদীকে একটি খাঁড়ি থেকে বড় হিসাবে সংজ্ঞায়িত করা হয়, [২] তবে সবসময় নয়; ভাষা অস্পষ্ট। [১]
অলিম্পিক উপদ্বীপের এলওয়া নদী
\par
আমাজন নদী (গাঢ় নীল) এবং নদীগুলি যা এর মধ্যে প্রবাহিত হয় (মাঝারি নীল)।
আথাবাস্কা হিমবাহ , জ্যাসপার ন্যাশনাল পার্ক , আলবার্টা, কানাডার গলিত পায়ের আঙুল
নদীগুলি জলচক্রের একটি গুরুত্বপূর্ণ অংশ । একটি নিষ্কাশন অববাহিকা থেকে জল সাধারণত বর্ষণ , প্রাকৃতিক বরফ এবং স্নোপ্যাক থেকে নির্গত গলিত জল এবং ভূগর্ভস্থ জল রিচার্জ এবং স্প্রিংসের মতো অন্যান্য ভূগর্ভস্থ উত্সগুলির মাধ্যমে একটি নদীতে জমা হয় ৷ নদীগুলিকে প্রায়শই একটি ল্যান্ডস্কেপের মধ্যে প্রধান বৈশিষ্ট্য হিসাবে বিবেচনা করা হয়; যাইহোক, তারা আসলে পৃথিবীর প্রায় 0.1\% ভূমি কভার করে। নদীগুলিও একটি গুরুত্বপূর্ণ প্রাকৃতিক টেরাফর্মার, কারণ প্রবাহিত জলের ক্ষয়কারী ক্রিয়াটি ভূপৃষ্ঠে রিল , গিরি এবং উপত্যকা তৈরি করে সেইসাথে পলি এবং দ্রবীভূত খনিজগুলিকে নীচের দিকে স্থানান্তর করে, নদী বদ্বীপ এবং দ্বীপগুলি গঠন করে যেখানে প্রবাহ হ্রাস পায়। জলাশয় হিসাবে, নদীগুলি জলজ এবং আধা জলজ প্রাণী এবং উদ্ভিদের জন্য , বিশেষত পরিযায়ী মাছের প্রজাতির জন্য মিঠা পানির আবাসস্থল প্রদান এবং খাওয়ানোর মাধ্যমে এবং সেইসাথে স্থলজগতের ইকোসিস্টেমগুলিকে নদী অঞ্চলে উন্নতি করতে সক্ষম করে গুরুত্বপূর্ণ পরিবেশগত কাজ করে ।
\par
নদীগুলি মানবজাতির জন্য তাৎপর্যপূর্ণ কারণ অনেক মানব বসতি এবং সভ্যতা বিশাল নদী এবং স্রোতের চারপাশে নির্মিত। [৩] বিশ্বের বেশিরভাগ প্রধান শহরগুলি নদীর তীরে অবস্থিত, কারণ তারা পানীয় জলের একটি অত্যাবশ্যক উত্স হিসাবে , মাছ ধরার এবং কৃষি সেচের মাধ্যমে খাদ্য সরবরাহের জন্য , শিপিংয়ের জন্য , প্রাকৃতিক হিসাবে নির্ভরশীল (বা ছিল)। সীমানা এবং/অথবা প্রতিরক্ষামূলক ভূখণ্ড, জলবিদ্যুতের উৎস হিসাবে যন্ত্রপাতি চালাতে বা বিদ্যুৎ উৎপন্ন করতে , স্নানের জন্য, এবং বর্জ্য নিষ্পত্তির উপায় হিসাবে । প্রাক-শিল্প যুগে , বৃহত্তর নদীগুলি অঞ্চল জুড়ে মানুষ, পণ্য এবং সেনাবাহিনীর চলাচলে একটি বড় বাধা ছিল । শহরগুলি প্রায়শই গড়ে ওঠে কয়েকটি স্থানে ফোর্ডিং , সেতু নির্মাণ বা সহায়ক বন্দরের জন্য উপযুক্ত ; অনেক বড় শহর, যেমন লন্ডন , সবচেয়ে সংকীর্ণ এবং সবচেয়ে নির্ভরযোগ্য স্থানে অবস্থিত যেখানে সেতু বা ফেরির মাধ্যমে নদী পার হওয়া যায় ।

আর্থ সায়েন্স ডিসিপ্লিনে , পটামোলজি হল নদীর বৈজ্ঞানিক অধ্যয়ন, যখন লিমনোলজি হল সাধারণভাবে অভ্যন্তরীণ জলের অধ্যয়ন।
একটি নদী একটি উৎস থেকে শুরু হয় (অথবা প্রায়শই অনেকগুলি উত্স) যা সাধারণত একটি জলাশয় হয়, তার নিষ্কাশন অববাহিকায় সমস্ত স্রোত নিষ্কাশন করে , একটি জলধারা অনুসরণ করে এবং একটি টার্মিনাসে শেষ হয় , হয় একটি সঙ্গম বা একটি মুখ বা মুখ দিয়ে, যা হতে পারে একটি নদী বদ্বীপ গঠন . একটি নদীর জল সাধারণত একটি চ্যানেলে সীমাবদ্ধ থাকে , তীরের মধ্যে একটি স্রোতের বিছানা দিয়ে তৈরি । বৃহত্তর নদীগুলিতে, প্রায়শই একটি প্রশস্ত প্লাবনভূমি থাকে যা বন্যার জল চ্যানেলের উপরে উঠে যায়। নদীপথের আকারের তুলনায় প্লাবনভূমি অনেক প্রশস্ত হতে পারে। নদী চ্যানেল এবং প্লাবনভূমির মধ্যে এই পার্থক্যটি অস্পষ্ট হতে পারে, বিশেষ করে শহুরে এলাকায় যেখানে একটি নদী চ্যানেলের প্লাবনভূমি আবাসন এবং শিল্প দ্বারা ব্যাপকভাবে বিকশিত হতে পারে।

" উপরিভার " এবং " ডাউনরিভার " শব্দগুলো যথাক্রমে নদীর উৎসের দিকে এবং নদীর মুখের দিকে নির্দেশ করে।
\end{document}