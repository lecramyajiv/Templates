%
% Copyright 2023 Lecram Yajiv

% Licensed under the Apache License, Version 2.0 (the "License");
% you may not use this file except in compliance with the License.
% You may obtain a copy of the License at

% http://www.apache.org/licenses/LICENSE-2.0

% Unless required by applicable law or agreed to in writing, software
% distributed under the License is distributed on an "AS IS" BASIS,
% WITHOUT WARRANTIES OR CONDITIONS OF ANY KIND, either express or implied.
% See the License for the specific language governing permissions and
% limitations under the License.

% translated from wikipedia about chera dynasty
% % https://en-m-wikipedia-org.translate.goog/wiki/Chera_dynasty?_x_tr_sl=auto&_x_tr_tl=ml&_x_tr_hl=en-GB

\RequirePackage[orthodox]{nag}
\documentclass[a4paper,12pt,oneside,final]{article}
\usepackage{microtype}
\usepackage[a4paper, margin=1.3cm, nohead, nofoot]{geometry}
\usepackage{fontspec}
\usepackage{setspace}
\pagestyle{empty}
\onehalfspacing
\usepackage[punjabi-guru, provide=*]{babel}
\babelfont[punjabi-guru]{rm}[Renderer=Harfbuzz]{Sant Lipi}
%\setmainfont[Script=Gurmukhi]{GUR Arjun}
%\newfontfamily\punjabifont{GUR Arjun}
%\DeclareTextFontCommand{\textpunjabi}{\punjabifont}
\begin{document}
ਬੈਂਜਾਮਿਨ ਫ਼ਰੈਂਕਲਿਨ ( 17 ਜਨਵਰੀ 1706 [ਪੁ.ਕ. 6 ਜਨਵਰੀ 1705] – 17 ਅਪਰੈਲ 1790) ਸੰਯੁਕਤ ਰਾਜ ਅਮਰੀਕਾ ਦੇ ਬਾਨੀ ਪਿਤਾਮਿਆਂ ਵਿੱਚੋਂ ਇੱਕ ਅਤੇ ਕਈ ਪੱਖਾਂ ਤੋਂ ਪਹਿਲਾ ਅਮਰੀਕੀ ਸੀ।[1] ਇੱਕ ਪ੍ਰਸਿੱਧ ਗਿਆਨਵਾਨ, ਫਰੈਂਕਲਿਨ ਇੱਕ ਪ੍ਰਮੁੱਖ ਲੇਖਕ ਅਤੇ ਪ੍ਰਿੰਟਰ, ਵਿਅੰਗਕਾਰ, ਰਾਜਨੀਤਕ ਚਿੰਤਕ, ਰਾਜਨੀਤੀਵਾਨ, ਵਿਗਿਆਨੀ, ਖੋਜੀ, ਸਿਵਲ ਸੇਵਕ, ਰਾਜਨੇਤਾ, ਫੌਜੀ, ਅਤੇ ਸਫ਼ਾਰਤੀ ਸੀ। ਇੱਕ ਵਿਗਿਆਨੀ ਦੇ ਰੂਪ ਵਿੱਚ, ਬਿਜਲੀ ਦੇ ਸੰਬੰਧ ਵਿੱਚ ਆਪਣੀ ਕਾਢਾਂ ਅਤੇ ਸਿਧਾਂਤਾਂ ਲਈ ਉਹ ਅਸਲੀ ਗਿਆਨ ਅਤੇ ਭੌਤਿਕ ਵਿਗਿਆਨ ਦੇ ਇਤਹਾਸ ਵਿੱਚ ਇੱਕ ਪ੍ਰਮੁੱਖ ਸ਼ਖਸੀਅਤ ਰਿਹਾ। ਉਸ ਨੇ ਬਿਜਲੀ ਦੀ ਛੜੀ, ਬਾਈਫੋਕਲਸ, ਫ਼ਰੈਂਕਲਿਨ ਸਟੋਵ, ਇੱਕ ਗੱਡੀ ਦੇ ਓਡੋਮੀਟਰ ਅਤੇ ਗਲਾਸ ਆਰਮੋਨਿਕਾ ਦੀ ਖੋਜ ਕੀਤੀ। ਉਸ ਨੇ ਅਮਰੀਕਾ ਵਿੱਚ ਪਹਿਲੀ ਪਬਲਿਕ ਕਰਜਾ ਲਾਇਬਰੇਰੀ ਅਤੇ ਪੈਨਸਿਲਵੇਨੀਆ ਵਿੱਚ ਪਹਿਲੇ ਅੱਗ ਵਿਭਾਗ ਦੀ ਸਥਾਪਨਾ ਕੀਤੀ। ਉਹ ਉਪਨਿਵੇਸ਼ਿਕ ਏਕਤਾ ਦੇ ਪਹਿਲੇ ਪ੍ਰਸਤਾਵਕਾਂ ਵਿਚੋਂ ਸੀ ਅਤੇ ਇੱਕ ਲੇਖਕ ਅਤੇ ਰਾਜਨੀਤਕ ਕਾਰਕੁੰਨ ਦੇ ਰੂਪ ਵਿੱਚ, ਉਸ ਨੇ ਇੱਕ ਅਮਰੀਕੀ ਰਾਸ਼ਟਰ ਦੇ ਵਿਚਾਰ ਦਾ ਸਮਰਥਨ ਕੀਤਾ। ਅਮਰੀਕੀ ਇਨਕਲਾਬ ਦੇ ਦੌਰਾਨ ਇੱਕ ਸਫ਼ਾਰਤੀ ਦੇ ਰੂਪ ਵਿੱਚ, ਉਸ ਨੇ ਫ਼ਰਾਂਸੀਸੀ ਜੋੜ-ਤੋੜ ਹਾਸਲ ਕੀਤਾ, ਜਿਸਨੇ ਅਮਰੀਕਾ ਦੀ ਆਜ਼ਾਦੀ ਨੂੰ ਸੰਭਵ ਬਣਾਉਣ ਵਿੱਚ ਮਦਦ ਕੀਤੀ।
\end{document}