%
% Copyright 2023 Lecram Yajiv

% Licensed under the Apache License, Version 2.0 (the "License");
% you may not use this file except in compliance with the License.
% You may obtain a copy of the License at

% http://www.apache.org/licenses/LICENSE-2.0

% Unless required by applicable law or agreed to in writing, software
% distributed under the License is distributed on an "AS IS" BASIS,
% WITHOUT WARRANTIES OR CONDITIONS OF ANY KIND, either express or implied.
% See the License for the specific language governing permissions and
% limitations under the License.

% translated from wikipedia about chera dynasty
% % https://en-m-wikipedia-org.translate.goog/wiki/Chera_dynasty?_x_tr_sl=auto&_x_tr_tl=ml&_x_tr_hl=en-GB

\RequirePackage[orthodox]{nag}
\documentclass[a4paper,12pt,oneside,final]{article}
\usepackage{microtype}
\usepackage[a4paper, margin=1.3cm, nohead, nofoot]{geometry}
\usepackage{fontspec}
\usepackage{setspace}
\pagestyle{empty}
\onehalfspacing
\usepackage[gujarati, provide=*]{babel}
\babelfont[gujarati]{rm}[Renderer=Harfbuzz]{Farsan}
%\setmainfont[Script=Gurmukhi]{GUR Arjun}
%\newfontfamily\punjabifont{GUR Arjun}
%\DeclareTextFontCommand{\textpunjabi}{\punjabifont}
\begin{document}
 ભારત દેશના ગુજરાત રાજ્યની ઇન્ડો-આર્યન ભાષા છે અને મુખ્યત્વે ગુજરાતી લોકો દ્વારા બોલાય છે. તે બૃહદ ઇન્ડો-યુરોપિયન ભાષા કુટુંબનો ભાગ છે. ગુજરાતીનો ઉદ્ભવ જૂની ગુજરાતી ભાષા (આશરે ઇ.સ. ૧૧૦૦-૧૫૦૦)માંથી થયો છે. તે ગુજરાત રાજ્ય અને દીવ, દમણ અને દાદરા-નગર હવેલી કેન્દ્રશાસિત પ્રદેશોની અધિકૃત ભાષા છે.

૨૦૧૧ની વસ્તી ગણતરી મુજબ, ભારતમાં વક્તાઓની સંખ્યા પ્રમાણે ગુજરાતી છઠ્ઠા ક્રમે સૌથી વધુ બોલાતી ભાષા છે. ૫.૫૬ કરોડ લોકો ગુજરાતી ભાષા બોલે છે, જે ભારતની વસ્તીના લગભગ ૪.૫\% જેટલા થાય છે.[૭] સમગ્ર વિશ્વમાં ગુજરાતી ભાષા બોલતા લોકોની સંખ્યા ૬.૫૫ કરોડ છે, જેથી ગુજરાતી ભાષા વિશ્વમાં ૨૦૦૭ મુજબ ૨૬મા ક્રમની સૌથી વધુ બોલાતી ભાષા છે.
\end{document}